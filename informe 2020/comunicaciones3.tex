\documentclass[10pt,a4paper]{article}
\usepackage[utf8]{inputenc}
\usepackage[spanish]{babel}
\usepackage{amsmath}
\usepackage{amsfonts}
\usepackage{amssymb}
\usepackage{graphicx}
\usepackage[left=2cm,right=2cm,top=2cm,bottom=2cm]{geometry}
\author{matias , matiasutn12@gmail.com \\ navarro \\amaya \\ }
\title{comunicaciones 3}
\date{} %sacar la fecha  
\usepackage{multicol}


\begin{document}
\maketitle  %ver el ttulo (y el autor)
\begin{multicols}{2} %2 es el numero de columnas

\begin{center}
\textbf{RESUMEN }
\end{center}
Este articulo presenta el estudio de los modos de operación de un convertidor reductor tipo Buck, los cuales son: modo de conducción discontinuo (DCM) y modo de conducción continuo (CCM). Se propone analizar mediante una simulación, los valores críticos de resistencia de carga que divide ambos modos. El software empleado para la simulación es el ORCAD PSpice
– This article presents the study of the operating modes of a Buck type reducer converter, which are: discontinuous driving mode (DCM) and continuous driving mode (CCM). It is proposed to analyze, by means of a simulation, the critical values of the load resistance that divides both modes. The software used for the simulation is the ORCAD PSpice.
– This article presents the study of the operating modes of a Buck type reducer converter, which are: discontinuous driving mode (DCM) and continuous driving mode (CCM). It is proposed to analyze, by means of a simulation, the critical values of the load resistance that divides both modes. The software used for the simulation is the ORCAD PSpice.

\begin{center}  %EMPIEZA LA INTRODUCCION 
\textbf{INTRODUCCION}

\end{center} %TERMINA LA INTRODUCION osea el titulo
iscontinuous driving mode (DCM) and continuous driving mode (CCM). It is proposed to analyze, by means of a simulation, the critical values of the load resistance that divides both modes. The software used for the simulation is the ORCAD PSpice.This article presents the study of the operating modes of a Buck type reducer converter, which are: discontinuous driving mode (DCM) and continuous driving mode (CCM). It is proposed to analyze, by means of a simulation, the critical values of the load resistance that divides both modes. The software used for the simulation is the ORCAD PSpice.This article presents the study of the operating modes of a Buck type reducer converter, which are: discontinuous driving mode (DCM) and continuous driving mode (CCM). It is proposed to 
$\alpha$ analyze, by means of a simulation, the critical  the ORCAD PSpice.

%saco la ecuacion de https://www.mathcha.io/editor
\begin{equation}
\frac{asd\ +\ asd\ }{asdasd}   
\end{equation}


– This article presents the study of the operating 
	\[c=\frac{qw}{rtrt} \dfrac{asd }{ere} \]  %tiene q terminae en \] 	
	\begin{equation}   %entre a ecuacion enumerada
	2^{45} 
	\end{equation}
	
	
\begin{center}
 \includegraphics[width=6cm]{C:/Users/matyas/Pictures/Imágenes guardadas/03 (3).jpg}
 \end{center} 



\begin{center}
Fig5 : matiasdasdasd
\end{center}
asdajfójo´fjasfasf


 
\begin{thebibliography}{99} %enumerar automatico numero de bibiografias  %referencias el titulo auto 

\bibitem{} {asdasdasdasdasdasdasd}

\end{thebibliography}


\end{multicols}
\end{document}
